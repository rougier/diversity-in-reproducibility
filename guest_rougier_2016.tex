\documentclass[jou]{apa6}
\usepackage[utf8]{inputenc}
\usepackage[english]{babel}
\usepackage{hyperref}
\usepackage{xcolor}
\hypersetup{
    colorlinks,
    linkcolor={red!50!black},
    citecolor={blue!50!black},
    urlcolor={blue!80!black}
}
\usepackage{apacite} 


% 500-1000 words by the 10th of January

\title{What is computational reproducibility?}
\shorttitle{What is computational reproducibility?}

\twoauthors{Olivia Guest}{Nicolas P. Rougier}
\twoaffiliations{Department of Experimental Psychology\\University College London, United Kingdom}{INRIA Bordeaux Sud-Ouest, Talence, France\\
Institut des Maladies Neurodégénératives, Université Bordeaux, Centre National de la Recherche Scientifique, UMR 5293, Bordeaux, France\\
LaBRI, Université de Bordeaux, Institut Polytechnique de Bordeaux, Centre National de la Recherche Scientifique, UMR 5800, Talence, France}
%olivia.guest@psy.ox.ac.uk
%nicolas.rougier@inria.fr
\abstract{}

\begin{document}
\maketitle
%Recap of previous 

In our previous contribution, we proposed definitions for \textbf{replicable} and \textbf{reproducible} in the context of computational modelling. We stressed the importance of comprehensive specification and of access to the original codebase. Furthermore, we highlighted an issue within scholarly communication: many journals do not require nor facilitate the sharing of code. In contrast, many free third-party services have filled the gaps left by traditional publishers \cite<e.g.,>{binder, github, osf, rescience}. However, journals and peers do not often require nor expect the use of these services. We ended by asking: is the scientific community is ready to associate codebases with articles; and are we prepared to ensure computational theories are well-specified and coherently implemented?


\bibliographystyle{apacite}
\bibliography{ref}
% The following space works around a bug in typesetting the references, where the hanging indent of the last reference is incorrectly set.
\hspace*{1cm}
\end{document}